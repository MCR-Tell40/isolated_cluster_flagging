% LHCb VELO upgrade presentation for 06 April 2017
% Topic:    Isolation flagging module progress
% Author:   Dónal Murray
% Date:     06 April 2017

% $Header$
\documentclass{beamer}
\mode<presentation>
\usefonttheme{professionalfonts}
\setbeamertemplate{footline}[text line]{%
  \parbox{\linewidth}{\vspace*{-22pt}\tiny\insertshortauthor}}
\setbeamertemplate{navigation symbols}{} % remove beamer symbols

\usepackage[english]{babel}
\usepackage[utf8]{inputenc}
\usepackage{times}
\usepackage[T1]{fontenc}
\usepackage{pgf}
\usepackage{tikz}

\usetikzlibrary{positioning,fit,calc}
\tikzstyle{block} = [draw, rectangle, minimum height=3em, minimum width=7em]
\tikzstyle{fifo} = [draw, rectangle, minimum height=8em, minimum width=7em]
\tikzstyle{bigblock} = [draw, rectangle, minimum height=8em, minimum width=10em]

\title[ICF Module]{Isolated Cluster Flagging Module}
\subtitle{Progress Update}
\author[Dónal Murray\hspace*{80pt}donal.murray@cern.ch]{Dónal Murray \\
  \vskip7pt
  \tiny{donal.murray@cern.ch}
}
\institute{}
\date{06 April 2017}

\pgfdeclareimage[height=1.5cm]{university-logo}{img/UoMlogo}
\logo{\pgfuseimage{university-logo}}

% $Document$
\begin{document}

{
\setbeamertemplate{footline}{} % do not display footer on titlepage
\begin{frame}
  \titlepage
\end{frame}
}
\addtocounter{framenumber}{-1} % do not count titlepage in slide count

\begin{frame}{Presentation Overview}
  \tableofcontents
\end{frame}


% Structuring a talk is a difficult task and the following structure
% may not be suitable. Here are some rules that apply for this
% solution:

% - Exactly two or three sections (other than the summary).
% - At *most* three subsections per section.
% - Talk about 30s to 2min per frame. So there should be between about
%   15 and 30 frames, all told.

% - A conference audience is likely to know very little of what you
%   are going to talk about. So *simplify*!
% - In a 20min talk, getting the main ideas across is hard
%   enough. Leave out details, even if it means being less precise than
%   you think necessary.
% - If you omit details that are vital to the proof/implementation,
%   just say so once. Everybody will be happy with that.

\section{Incorporation into the full AMC40 firmware}

\begin{frame}{Incorporation into the full AMC40 Firmware}
  \begin{itemize}
    \item
      Cloned the updated VELO24 branch
      \begin{itemize}
        \item
          vhd file containing entity definition missing - the file is RAM\_ID\_DEMUX.vhd
          in the specific velopix directory
        \item
          Copied over file again and compiled without any
        \item
          Would like to check this file is correct
      \end{itemize}
    \item
      Currently redesigning the module to have constant timing
    \end{itemize}
\end{frame}

\subsection{Position in firmware}

\begin{frame}{Position in firmware}
  \begin{itemize}
    \item
      Data processing block has changed since this module was designed
      \begin{itemize}
        \item
          Original positioning needed to be changed because of the edge detectors in the post router
        \item
          The new position will be right at the start of the post router before the edge detectors
        \item
          The flagging bit is the MSB of each 32bit word, therefore taking input from inflactionary\_block
      \end{itemize}
    \end{itemize}
\end{frame}

\subsection{Timing}

\begin{frame}{Timing}
  \begin{itemize}
    \item
      Each data processor within the ICF block needs the following number of clock cycles per BCID:
      \begin{itemize}
        \item
            1 clock cycle to check if it is empty (0 SPPs)
        \item
            4 clock cycles to read in the 4 512bit frames associated with the BCID
        \item
            65 clock cycles to sort the columns and flag isolated clusters
        \item
            4 clock cycles to write it out
      \end{itemize}
    \item
      The BCID processing is parallelised within the ICF module
    \item
      16 rams could be processed every 229 clock cycles with 16 ICF modules
  \end{itemize}
\end{frame}


\section*{Summary}

\begin{frame}{Summary}

  % Keep the summary as short as possible
  \begin{itemize}
  \item
    New design to fit in with the post router
  \item
    Timing now constant in new design
  \item
    Testing new design in Modelsim
  \end{itemize}

  % Next steps
  \vskip0pt plus.5fill
  \begin{itemize}
  \item
    Outlook
    \begin{itemize}
    \item
      Test the new top level in Modelsim
    \item
      Incorporate into the post router as a drop in module
    \end{itemize}
  \end{itemize}
\end{frame}

\end{document}
