% LHCb VELO upgrade presentation for 23 March 2017
% Topic:    Isolation flagging module progress
% Author:   Dónal Murray
% Date:     21 March 2017

% $Header$
\documentclass{beamer}
\mode<presentation>
\usefonttheme{professionalfonts}
\setbeamertemplate{footline}[text line]{%
  \parbox{\linewidth}{\vspace*{-22pt}\tiny\insertshortauthor}}
\setbeamertemplate{navigation symbols}{} % remove beamer symbols

\usepackage[english]{babel}
\usepackage[utf8]{inputenc}
\usepackage{times}
\usepackage[T1]{fontenc}
\usepackage{pgf}

\title{Event Isolation Flagging Module}
\subtitle{Progress Update}
\author[Dónal Murray\hspace*{80pt}donal.murray@cern.ch]{Dónal Murray \\
  \vskip7pt
  \tiny{donal.murray@cern.ch}
}
\institute{}
\date{23 March 2017}

\pgfdeclareimage[height=1.5cm]{university-logo}{UoMlogo}
\logo{\pgfuseimage{university-logo}}

% $Document$
\begin{document}

{
\setbeamertemplate{footline}{} % do not display footer on titlepage
\begin{frame}
  \titlepage
\end{frame}
}
\addtocounter{framenumber}{-1} % do not count titlepage in slide count

\begin{frame}{Overview}
  \tableofcontents
\end{frame}


% Structuring a talk is a difficult task and the following structure
% may not be suitable. Here are some rules that apply for this
% solution:

% - Exactly two or three sections (other than the summary).
% - At *most* three subsections per section.
% - Talk about 30s to 2min per frame. So there should be between about
%   15 and 30 frames, all told.

% - A conference audience is likely to know very little of what you
%   are going to talk about. So *simplify*!
% - In a 20min talk, getting the main ideas across is hard
%   enough. Leave out details, even if it means being less precise than
%   you think necessary.
% - If you omit details that are vital to the proof/implementation,
%   just say so once. Everybody will be happy with that.

\section{Concept}

\subsection{Function of the event isolation flagging module}

\begin{frame}{The event isolation flagging module}
  \begin{itemize}
  \item
    Sorts columns in SPP
  \item
    Checks each column to see if it is isolated
  \item
    Either adds a flag bit or bypasses it
  \end{itemize}
\end{frame}



\subsection{Flow diagrams}

\begin{frame}{Flow diagrams}{Top level}
\end{frame}

\begin{frame}{Flow diagrams}{Sorter}
\end{frame}


\subsection{Block diagrams}

\begin{frame}{Block diagrams}{Top level}
\end{frame}

\begin{frame}{Block diagrams}{Active controller}
\end{frame}

\begin{frame}{Block diagrams}{Data processor}
\end{frame}



\section{Implementation}

\subsection{Implementation in VHDL}

\begin{frame}{Implementation in VHDL}
\end{frame}


\subsection{Testing in Modelsim}

\begin{frame}{Testing in Modelsim}
\end{frame}


\subsection{Incorporation into the full AMC40 Firmware}

\begin{frame}{Incorporation into the full AMC40 Firmware}
\end{frame}



\section*{Summary}

\begin{frame}{Summary}

  % Keep the summary as short as possible
  \begin{itemize}
  \item
    Implementation in Modelsim is complete
  \item
    Mid way through testing
  \end{itemize}

  % Next steps
  \vskip0pt plus.5fill
  \begin{itemize}
  \item
    Outlook
    \begin{itemize}
    \item
      Complete testing in Modelsim
    \item
      Test as a standalone module in Quartus
    \item
      Incorporate into full AMC40 firmware
    \end{itemize}
  \end{itemize}
\end{frame}

\end{document}
